%
%  Vorlage/Template fuer #Proseminar
%
%  Created by Claudia Pittel und Yunus Erdemir in 2018, updated in 2021.
%  Copyright (c) 2021 . All rights reserved.
%
\documentclass[12pt,toc=bib,toc=listof]{scrreprt}
\usepackage[ngerman]{babel} 
\usepackage[utf8]{inputenc}
\usepackage[T1]{fontenc}
% Standardschriftart Latin Modern
\usepackage[scaled]{helvet}
\renewcommand\familydefault{\sfdefault}
% Arial, so in etwa
%\usepackage{newtxmath,newtxtext}
%Times New Roman, so in etwa
% Zitate durchgehend nummerieren
\usepackage{chngcntr}
\counterwithout{footnote}{chapter}
%Zeilenabstand
\usepackage{setspace}
\linespread{1.3}

% DIN-A-4, Ränder (oben, links, rechts, unten), Abstand Kopfzeile und Text
\usepackage{geometry}
\geometry{a4paper, top=25mm, left=30mm, right=20mm, bottom=25mm, headsep=12mm}

\usepackage[backend=bibtex, style=authortitle-ibid, url=false, isbn=false, doi=false]{biblatex}
\addbibresource{literature.bib}

\usepackage{hyperref}
\hypersetup{
  ,colorlinks=true
  ,linkcolor=blue
  ,citecolor=blue
  ,filecolor=blue
  ,urlcolor=blue
  }

%%%%%%%%%%%%%%%%%%%%%%%%%%%%%%%%%%%%% % (fold)
% Vom Studierenden zu aendernde Werte
\newcommand{\topic}{XaaS - Anything as a Service}

\newcommand{\studentnameA}{Suphi Pembe}
\newcommand{\studentidA}{207617}
\newcommand{\studentpartA}{SEITEN oder KAPITEL VON BIS 1}

\newcommand{\studentnameB}{Andreas Würzer}
\newcommand{\studentidB}{207258}
\newcommand{\studentpartB}{SEITEN oder KAPITEL VON BIS 2}

\newcommand{\studentnameC}{Christian Nguyen}
\newcommand{\studentidC}{207613}
\newcommand{\studentpartC}{SEITEN oder KAPITEL VON BIS 3}

\newcommand{\semester}{Sommersemester 2022}
%
%%%%%%%%%%%%%%%%%%%%%%%%%%%%%%%%%%%%% % (end)

\usepackage{ifpdf}
\ifpdf
\usepackage[pdftex]{graphicx}
\else
\usepackage{graphicx}
\fi

\usepackage[headsepline,footsepline]{scrlayer-scrpage}
\pagestyle{scrheadings}
\clearpairofpagestyles
\ihead{Proseminar: \topic}
\ofoot{Seite \pagemark}
\renewcommand*{\chapterpagestyle}{scrheadings}
\renewcommand*{\chapterheadstartvskip}{}

% Logo
\titlehead{\flushright\includegraphics[scale=0.5]{HHN_Logo_D_oS_RGB_300.png}}
\subject{Proseminar (282136)}
\title{\topic}
\author{\studentnameA { ({\studentidA)}}, \\ \studentnameB { ({\studentidB)}}, \\ \studentnameC { ({\studentidC)}} }
\date {\semester}
%% Datum nie auf einen festen Wert setzen
\publishers{Vorgelegt bei Claudia Pittel}

%\pagestyle{headings}

% Zähler für Römische Nummerierung
\newcounter{savepage}

\begin{document}
\pagenumbering{roman} 
\selectlanguage{ngerman}
\sffamily
% serifenlose Schriftfamilie
\maketitle

\addchap{Management Summary} % (fold)
\label{sec:management_summary}

Hier sollte ziemlich genau bzw. maximal 1 Seite Text stehen (ziemlich genau  bedeutet, man sollte so nah wie möglich an 1 Seite herankommen).
Text für Test commit haha

% chapter management_summary (end)

\tableofcontents

\addchap{Abkürzungsverzeichnis} % (fold)
\label{sec:abkuerzungsverzeichnis}

\begin{description}
\item[GPU] Graphics-Processing-Unit oder Grafikkarte
\item[HPC] High-Performance-Computing
\end{description}

% chapter abkuerzungsverzeichnis (end)

\listoffigures
\listoftables

% \onehalfspacing

\newpage
% Zähler speichern
\setcounter{savepage}{\arabic{page}}
\pagenumbering{arabic}

\chapter{Einleitung} % (fold)
\label{sec:einleitung}

Einleitungstext mit Motivation, Ziel der Arbeit (d.h. Erläuterung der Forschungsfrage) und Beschreibung der Vorgehensweise bzw. Aufbau der Arbeit\footcite [vgl.] [S. 38]{Th17}

\section{Motivation} % (fold)
\label{sec:motivation}
%%%% Version 1 %%%%%%%%
Durch den aktuell anhaltende Halbleitermangel besteht ein Engpass an Ressourcen von die meisten Wirtschaftszweige betroffen sind.
Einer dieser Wirtschaftszweige ist die Produktion von GPUs (graphics processing unit. Diese werden für diverse Anwendung von Computern verwendet, im betrieblichen wie auch im privaten Bereich.
Primär in dieser Arbeit werden die Bereiche High-Performance-Computing (HPC) und Gaming haben.
Beide diese Bereiche benötigen GPU-Rechenleistung, welche im konventionell von einer lokal verbauten GPU zur Verfügung gestellt wird. 
Als Langfristige Lösung soll analysiert werden ob es möglich ist durch zentrale Services, welche GPU as a Service anbieten. 
Durch die zentralen Ressourcenteilung dem Mangel entgegenzuwirken mit einer alternative für den Bedarf zu schaffen.

% section motivation (end)

\section{Ziel der Arbeit} % (fold)
\label{sec:ziel_der_arbeit}

\ldots

% section ziel_der_arbeit (end)

\section{Vorgehensweise} % (fold)
\label{sec:vorgehensweise}

\ldots

% section vorgehensweise (end)
% chapter einleitung (end)

\chapter{Anything as a Service - Cloud Computing} % (fold)
\label{sec:Anything as a Service - Cloud Computing}

Zwischen den Gliederungspunkten sollten jeweils kurze Überleitungssätze stehen, damit man weiß, um was es inhaltlich in den folgenden Unterkapiteln geht.\footcite [Vgl.] []{GPL}

% chapter ersteskapitel (end)

\section{Definition} % (fold)
\label{sec:Definition}

Bei den Gliederungspunkten immer auf eine Ausgewogenheit achten, damit eine gleichmäßige Gliederung gefördert werden kann. Sofern Abbildungen (wie Abbildung 1: Beispielbild) verwendet werden, müssen diese auch inhaltlich im Text erwähnt und erläutert werden, sowie ein Abbildungsverzeichnis erstellt werden.\footcite [Vgl.] [] {hhnwin}

% section unterkapitel1 (end)

\section{Typische Servicemodelle} % (fold)
\label{sec:Typische Servicemodelle}

Untergliederungen nur in der Mehrzahl erstellen, d.h. nie 1 Unterkapitel alleine stehen lassen.
In gleicher Art und Weise wie Abbildungen dargestellt und beschriftet werden, verhält es sich mit Tabellen.

% section unterkapitel1 (end)

\subsection{IaaS: Infrastructure as a Service}
\label{sec:IaaS: Infrastructure as a Service}

Inhalt

\subsection{SaaS: Software as a Service}
\label{sec:SaaS: Software as a Service}

Inhalt

\subsection{PaaS: Plattform as a Service}
\label{sec:PaaS: Plattform as a Service}


\section{Vor- und Nachteile}
\label{Vor- und Nachteile}


\chapter{Knappheit von Grafikkarten} % (fold)
\label{sec:Knappheit von Grafikkarten}

Die Knappheit von Grafikkarten, in Zukunft als GPU bezeichnet, hat den
aktuellen Markt durch neue Branchen die GPU-Leistung nutzen nachhaltig verändert.
Diese Knappheit entsteht nicht nur durch den Mangel des Rohstoffes, sondern auch durch die 
weiterentwicklung von verwendeten Computern in allen Einsatzgebieten.\footcite [Vgl.] []{Voas.2021}
\\
%\begin{figure}[h]
%  \centering
%  \includegraphics[scale=0.4]{Abbildungen/Martin_Kords_1.png} % 0,9 ursprünglich, auf Seite anpassen
%  \caption[]{Weltweite Lieferung von Halbleiterprodukten für die Automobilindustrie von 2011 bis 2021}
%\end{figure}
\\
Im Vergleich zu 2011 wurden für die Automobilindustrie im 2021 fast drei mal so viele
Halbleiter geliefert. Ebensfalls mit der weiterentwicklung von internet of things Produkten wird In
Zukunft der Bedarf an Halbleitern weiter steigen.\footcite [Vgl.] []{Bill_McClean} \footcite [Vgl.] []{Voas.2021} 
%\footcite [Vgl. Bill McClean "The 2022 McClean Report"]
%Quelle https://www.icinsights.com/news/bulletins/The-Real-Reason-Behind-The-Automotive-Industry-IC-ShortageA-StepFunction-Surge-In-Demand/
\\In diesem Kapitel soll die Preisentwicklung von GPUs betrachtet werden, dabei wird 
ein Zusammenhang geschaffen mit den Ursachen die diese Preisentwicklung 
verursacht haben.
\\

\section{Preisentwicklung}
\label{sec:Preisentwicklung}

Die rapide steigende Preisentwicklung von GPUs ist auf zwei Kernfaktoren reduzierbar.\\
- Größerer Bedarf an GPUs und Halbleitern, dem Kernbestandteil von GPUs\\
- Mangelnde Kapazitäten zur Produktion von Halbleitern\\
Der Bedarf an Halbleitern und GPUs ist konstant im Anstieg. Besonders durch die Corona Pandemie,
hat sich im Vergleich zu 2019 im Jahr 2020 ein Umsatzanstieg von 5,4\%.\footcite [Vgl.] []{Voas.2021}
\begin{figure}[h]
  \centering
  \includegraphics[scale=0.5]{Abbildungen/voas1.png} % 0,9 ursprünglich, auf Seite anpassen
  \caption[]{Worldwide semiconductor revenues in 2019 and 2020 (dollar, billions)}
\end{figure}
%%%%%%%% GRAFIKBEZEICHNUNG!!!!!!! %%%%%%%%%%%%%%%%%%
\\Wie in Abbildung 3.X zu sehen ist der Umsatzanstieg größtenteils durch Erlöse von Computersystemen enstanden.
Im Vergleich dazu sind Umsätze die durch Abnehmer in der Automobielbranche enstanden sind gesunken.
Das lässt sich auf den steigenden Bedarf an Computersystemen zurückführen. Währrend der Pandemie mussten
viele Menschen Home-Schooling und Home-Office aneignen um weiter den Alltagsbetrieb ausführen zu können.
Ein Nebenläufiger Effekt ist damit, dass die Leute durch die digitalisierung weniger mobilität benötigen.
Damit lässt sich der geringe Bedarf an Halbleitern erklären. Dennoch ist damit insgesamt der Bedarf an 
Halbleiter gestiegen.\footcite [Vgl.] []{Voas.2021}



\section{Ursache Halbleitermangel und KryptoMining}
\label{sec:Ursache Halbleitermangel und KryptoMining}

Inhalt

\chapter{Gaming as a Service}
\label{sec: Gaming as a Service}

Cloud Computing umfasst die Bereitstellung von Rechenleistung und Anwendungen als Dienst über das Internet und soll daher mit Gaming-as-a-Service als Beispiel vertieft werden. Neben der Funktionsweise, werden auch die aktuell verfügbaren Angebote verglichen und damit dem Kauf eines eigenen Computers gegenübergestellt.

\section{Funktionsweise}
\label{sec:Funktionsweise}

Die Ausführung der Spiele, einschließlich der Spielelogik und Wiedergabe der Szenen findet innerhalb der Cloud bzw. Server statt. In Verbindung steht das Gerät des Endnutzers, oder auch Thin-Client genannt. Dieser empfängt die komprimiert gestreamten Audio- und Videosignale über das Internet und gibt sie auf dem Thin-Client wieder. Bei eingehenden Befehlen des Endnutzern, werden diese erfasst und an die Cloud übertragen. Durch die Leistung des Netzwerks zwischen dem Client und der Cloud sind die Prozesse eingeschränkt.

\section{Anbietervergleich}
\label{sec:Anbietervergleich}

Inhalt

\subsection{Voraussetzung}
\label{sec:Vorraussetzung}

Inhalt

\subsection{Angebot}
\label{sec:Angebot}

Inhalt

\subsection{Preis}
\label{sec:Preis}

Inhalt

\section{Hardwarevorraussetzung um Usability zu gewährleisten}
\label{sec:Hardwarevorraussetzung um Usability zu gewährleisten}

Inhalt

\chapter{GPU as a Service}
\label{sec:GPU as a Service}

Inhalt

\section{Funktionsweise}
\label{sec: Funktionsweise}

Inhalt

\section{Einsatzgebiete}
\label{sec: Einsatzgebiete}

Inhalte

\section{Vergleich eigene GPU und GPU in der Cloud}
\label{sec:Vergleich eigene GPU und GPU in der Cloud}

Inhalt

\chapter{Marktvorhersage}
\label{sec:Marktvorhersage}

Inhalt

\chapter{Fazit und Ausblick} % (fold)
\label{sec:fazit}

Kritische Begutachtung inklusive Zusammenfassung der Arbeit sowie eventuelle Zukunftsperspektiven zum Thema können hier im Fazit und im Ausblick eingebracht werden.\footcite [Vgl.] [] {HAN2016S30}

% chapter fazit (end)

\appendix
\newpage

\pagenumbering{roman}
% Zähler laden
\setcounter{page}{\thesavepage}

\addchap{Anhang} % (fold)
\label{sec:anhang}

Der Anhang soll den eigentlichen Hauptteil nicht ergänzen, sondern darüber hinaus weitere möglicherweise interessante Informationen liefern, die aber nicht zwangsläufig notwendig sind, um den Hauptinhalt zu verstehen

\newpage

% Quellen wird erst dargestellt sobald ein Zitat verwendet wird
\defbibheading{head}{\addchap{Quellenverzeichnis}}
\printbibliography[heading=head]

\vspace{2cm}

Hier müssen alle Quellenverweise zu finden sein – inklusive aller erforderlichen Angaben, alphabetisch sortiert. Eine Unterteilung in verschiedene Quellenarten ist grundsätzlich nicht notwendig, da die unterschiedlichen Quellenarten anhand der Angabe der bibliographischen Angaben zu erkennen ist. (d.h. beispielsweise keine Unterteilung zwischen „Printquellen“ und „Internetquellen“!)

\newpage

\addchap{Ehrenwörtliche Erklärung} % (fold)
\label{sec:erklaerung}

„Wir versichern, dass die vorliegende Arbeit von uns selbständig und ausschließlich unter Verwendung der angegebenen Quellen und Hilfsmittel angefertigt wurde. Alle Stellen, die wörtlich oder annähernd aus Veröffentlichungen entnommen sind, haben wir als solche kenntlich gemacht. Die Arbeit wurde bisher in gleicher oder ähnlicher Form, auch nicht in Teilen, keiner anderen Prüfungsbehörde vorgelegt und auch nicht veröffentlicht.“

\vspace{1cm}
\noindent
{\studentpartA} wurden von {\studentnameA} verfasst.
\\
{\studentpartB} wurden von {\studentnameB} verfasst.
\\
{\studentpartC} wurden von {\studentnameC} verfasst.

\vspace{3cm}
Ort, Datum \hfill Unterschrift

\vspace{2cm}
Ort, Datum \hfill Unterschrift

\vspace{2cm}
Ort, Datum \hfill Unterschrift


\end{document}
